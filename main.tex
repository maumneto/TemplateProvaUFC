%%%%%%%%%%%%%%%%%%%%%%%%%%%%%%%%%%%%%%%%%%%%%%%%%%%%%%%%%%%%%%%%%%%%%%%%%%%%%%%%%%
%% This project aims to create a test template or exercise list from the        %%     
%% Federal University of Ceará (UFC).                                           %%
%% author: Maurício Moreira Neto - Doctoral student in Computer Science         %%
%% contacts:                                                                    %%
%%    e-mail: maumneto@ufc.br                                                   %%
%%    linktree: https://linktr.ee/maumneto                                     %%
%%%%%%%%%%%%%%%%%%%%%%%%%%%%%%%%%%%%%%%%%%%%%%%%%%%%%%%%%%%%%%%%%%%%%%%%%%%%%%%%%%
\documentclass[11pt, a4paper, answers]{exam}

\usepackage{documentufc}
\usepackage[portuguese]{babel}
\usepackage{amsmath}
\usepackage{amssymb}
\usepackage[utf8]{inputenc}

%% Informations that will be insert in the table header 
\def\course{Nome da Disciplina}
\def\prof{Nome do Autor}
\def\semester{XXXX.X}
\def\codeCourse{XXXXXX}
\def\registration{}
\def\student{}
\def\graduate{Nome do Curso}
\def\theme{Descrição do Tema
}

\begin{document}
    %% Table with the header
    \makeheader
    
    %% Space for the instructions
    \fbox{
        \parbox{\textwidth}{
            \begin{minipage}{\textwidth}
                \makeinstructions
                {
                    \begin{instlist}
                        \item A avaliação é individual e não é pesquisada.
                        \item Preencha o cabeçalho da folha pergunta com seus dados.
                        \item  Todas as folhas respostas devem conter o nome a a matrícula do aluno.
                        \item O preenchimento das respostas deve ser feito utilizando caneta (preta ou azul).
                    \end{instlist}
                }
            \end{minipage}
        }
    }
    %% Space between the instructions and the questions.
    \vspace{1cm}
    
    \begin{question}
        \item \potuation{2}. As questões podem ser elencadas usando os comandos \textit{begin} e \textit{end} com o argumento question. Desta forma, cada questão é referente a um \textit{item} da estrutura \textit{question}.

        \item \potuation{3.5}. Outra funcionalidade interessante: caso você queira colocar pontuação em cada questão, basta colocar o comando \textit{potuation} com o valor da pontuação da questão. 
        
        \item \potuation{4}. A ideia é que este template seja sempre atualizado, visando atender a todas as necessidades dos usuários da UFC. Caso possua alguma dica que melhore o template, é possível entrar em contato através do e-mail disposto nos comentários.
        
        \item \potuation{4}. Também é possível criar questões fechadas de multiplas escolhas usando o \LaTeX. Para isso basta utilizar o comando \textit{multiplechoices}. É necessário utilizar a quebra de linhas nessas questões. \\
        \begin{multiplechoices}
            \choice Item 1
            \choice Item 2
            \choice Item 3
            \choice Item 3
        \end{multiplechoices}
    \end{question}
\end{document}